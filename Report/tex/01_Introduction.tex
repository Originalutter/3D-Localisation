
\section{Introduction}
	\subsection{Background}
		Localisation of objects has been of importance for the society for a long time.  Today \gls{gps} can take us within metres of certainty \cite{_gps.gov:_2015}. When locating the position of a building, this level of accuracy is acceptable. However, when it comes to localisation of objects indoors, when there is need to know where inside of a building something is situated, it is not sufficient. The location of an object in 3-D could for example be useful in large warehouses or in emergency situations.

		Several studies have proposed an Indoor Positioning System (IPS) that utilises Received Signal Strength Indication (RSSI) for 3-D localisation of indoor objects
		\cite{lee_3d-localization---mobile-beacon.pdf_2012,kim_mobile_2010,hassan_indoor_2010}. RSSI is a measurement of the power present in a received radio signal and is used to estimate the distance between the transmitter and the target. The distance is needed to calculate the approximated location of the target, by applying three-dimensional trilateration. While RSSI based IPS's have been shown to successfully approximate the location in 3-D, the average accuracy ranges from 1 to 2 metres \cite{hassan_indoor_2010,luo_comparative_2011}. Taken into account that the ceiling height in houses are circa 2.4 metres, an accuracy of 2 metres is not sufficient for localisation of the targets vertical position in the room.

		Bharadwaj proposes an IPS using \gls{uwb} technology for sensor communication
		\cite{bharadwaj_ultrawideband-based_2014}, which allows high-resolution positioning by high-precision ranging based on \gls{toa}. \gls{toa} is the travel time of a radio signal from a transmitter to a receiver. As radio signals travel with known velocity, \gls{toa} can be used to calculate the distance between the sensors. Once the distance is calculated, trilateration can be used in the same way as the RSSI-based IPS. A working example that uses \gls{toa}, and works similarly to Bharadwaj's proposed IPS, is \gls{gps}. A difference with \gls{gps} is that it has to take the Special and General theories of Relativity \cite{einstein_relativity_1920} into account, since the transmitters (satellites) are constantly moving \cite{_gps_????} as opposed to Bharadwaj's proposed IPS. \gls{uwb} based IPS's have shown significantly improved accuracy compared to RSSI based IPS's, reaching centimetre\cite{bharadwaj_ultrawideband-based_2014} and even millimetre\cite{zhang_real-time_2009} precision.

		%State your contributions 
		This report investigates the UWB based IPS further, by confirming previous work and expanding the test environment to a full-sized furnished room. It further evaluates the system's practical use and robustness regarding interference from other objects.

		%Research question(s)
		\subsection{Aim}
		The general aim of this project was to investigate if \gls{uwb} based positioning systems are suitable for 3-D localisation of objects in furnished rooms.
		To answer this, the following questions were asked:
		\begin{itemize}	
			\item How accurately can a \gls{uwb} based positioning system localise an object in 3-D?
			\item How sensitive is a \gls{uwb} based positioning system to interference?
		\end{itemize}

		%Related work
	%\subsection{Related Work}

\clearpage